% Options for packages loaded elsewhere
\PassOptionsToPackage{unicode}{hyperref}
\PassOptionsToPackage{hyphens}{url}
%
\documentclass[
]{article}
\title{regression.R}
\author{Diaraye}
\date{2022-06-02}

\usepackage{amsmath,amssymb}
\usepackage{lmodern}
\usepackage{iftex}
\ifPDFTeX
  \usepackage[T1]{fontenc}
  \usepackage[utf8]{inputenc}
  \usepackage{textcomp} % provide euro and other symbols
\else % if luatex or xetex
  \usepackage{unicode-math}
  \defaultfontfeatures{Scale=MatchLowercase}
  \defaultfontfeatures[\rmfamily]{Ligatures=TeX,Scale=1}
\fi
% Use upquote if available, for straight quotes in verbatim environments
\IfFileExists{upquote.sty}{\usepackage{upquote}}{}
\IfFileExists{microtype.sty}{% use microtype if available
  \usepackage[]{microtype}
  \UseMicrotypeSet[protrusion]{basicmath} % disable protrusion for tt fonts
}{}
\makeatletter
\@ifundefined{KOMAClassName}{% if non-KOMA class
  \IfFileExists{parskip.sty}{%
    \usepackage{parskip}
  }{% else
    \setlength{\parindent}{0pt}
    \setlength{\parskip}{6pt plus 2pt minus 1pt}}
}{% if KOMA class
  \KOMAoptions{parskip=half}}
\makeatother
\usepackage{xcolor}
\IfFileExists{xurl.sty}{\usepackage{xurl}}{} % add URL line breaks if available
\IfFileExists{bookmark.sty}{\usepackage{bookmark}}{\usepackage{hyperref}}
\hypersetup{
  pdftitle={regression.R},
  pdfauthor={Diaraye},
  hidelinks,
  pdfcreator={LaTeX via pandoc}}
\urlstyle{same} % disable monospaced font for URLs
\usepackage[margin=1in]{geometry}
\usepackage{color}
\usepackage{fancyvrb}
\newcommand{\VerbBar}{|}
\newcommand{\VERB}{\Verb[commandchars=\\\{\}]}
\DefineVerbatimEnvironment{Highlighting}{Verbatim}{commandchars=\\\{\}}
% Add ',fontsize=\small' for more characters per line
\usepackage{framed}
\definecolor{shadecolor}{RGB}{248,248,248}
\newenvironment{Shaded}{\begin{snugshade}}{\end{snugshade}}
\newcommand{\AlertTok}[1]{\textcolor[rgb]{0.94,0.16,0.16}{#1}}
\newcommand{\AnnotationTok}[1]{\textcolor[rgb]{0.56,0.35,0.01}{\textbf{\textit{#1}}}}
\newcommand{\AttributeTok}[1]{\textcolor[rgb]{0.77,0.63,0.00}{#1}}
\newcommand{\BaseNTok}[1]{\textcolor[rgb]{0.00,0.00,0.81}{#1}}
\newcommand{\BuiltInTok}[1]{#1}
\newcommand{\CharTok}[1]{\textcolor[rgb]{0.31,0.60,0.02}{#1}}
\newcommand{\CommentTok}[1]{\textcolor[rgb]{0.56,0.35,0.01}{\textit{#1}}}
\newcommand{\CommentVarTok}[1]{\textcolor[rgb]{0.56,0.35,0.01}{\textbf{\textit{#1}}}}
\newcommand{\ConstantTok}[1]{\textcolor[rgb]{0.00,0.00,0.00}{#1}}
\newcommand{\ControlFlowTok}[1]{\textcolor[rgb]{0.13,0.29,0.53}{\textbf{#1}}}
\newcommand{\DataTypeTok}[1]{\textcolor[rgb]{0.13,0.29,0.53}{#1}}
\newcommand{\DecValTok}[1]{\textcolor[rgb]{0.00,0.00,0.81}{#1}}
\newcommand{\DocumentationTok}[1]{\textcolor[rgb]{0.56,0.35,0.01}{\textbf{\textit{#1}}}}
\newcommand{\ErrorTok}[1]{\textcolor[rgb]{0.64,0.00,0.00}{\textbf{#1}}}
\newcommand{\ExtensionTok}[1]{#1}
\newcommand{\FloatTok}[1]{\textcolor[rgb]{0.00,0.00,0.81}{#1}}
\newcommand{\FunctionTok}[1]{\textcolor[rgb]{0.00,0.00,0.00}{#1}}
\newcommand{\ImportTok}[1]{#1}
\newcommand{\InformationTok}[1]{\textcolor[rgb]{0.56,0.35,0.01}{\textbf{\textit{#1}}}}
\newcommand{\KeywordTok}[1]{\textcolor[rgb]{0.13,0.29,0.53}{\textbf{#1}}}
\newcommand{\NormalTok}[1]{#1}
\newcommand{\OperatorTok}[1]{\textcolor[rgb]{0.81,0.36,0.00}{\textbf{#1}}}
\newcommand{\OtherTok}[1]{\textcolor[rgb]{0.56,0.35,0.01}{#1}}
\newcommand{\PreprocessorTok}[1]{\textcolor[rgb]{0.56,0.35,0.01}{\textit{#1}}}
\newcommand{\RegionMarkerTok}[1]{#1}
\newcommand{\SpecialCharTok}[1]{\textcolor[rgb]{0.00,0.00,0.00}{#1}}
\newcommand{\SpecialStringTok}[1]{\textcolor[rgb]{0.31,0.60,0.02}{#1}}
\newcommand{\StringTok}[1]{\textcolor[rgb]{0.31,0.60,0.02}{#1}}
\newcommand{\VariableTok}[1]{\textcolor[rgb]{0.00,0.00,0.00}{#1}}
\newcommand{\VerbatimStringTok}[1]{\textcolor[rgb]{0.31,0.60,0.02}{#1}}
\newcommand{\WarningTok}[1]{\textcolor[rgb]{0.56,0.35,0.01}{\textbf{\textit{#1}}}}
\usepackage{graphicx}
\makeatletter
\def\maxwidth{\ifdim\Gin@nat@width>\linewidth\linewidth\else\Gin@nat@width\fi}
\def\maxheight{\ifdim\Gin@nat@height>\textheight\textheight\else\Gin@nat@height\fi}
\makeatother
% Scale images if necessary, so that they will not overflow the page
% margins by default, and it is still possible to overwrite the defaults
% using explicit options in \includegraphics[width, height, ...]{}
\setkeys{Gin}{width=\maxwidth,height=\maxheight,keepaspectratio}
% Set default figure placement to htbp
\makeatletter
\def\fps@figure{htbp}
\makeatother
\setlength{\emergencystretch}{3em} % prevent overfull lines
\providecommand{\tightlist}{%
  \setlength{\itemsep}{0pt}\setlength{\parskip}{0pt}}
\setcounter{secnumdepth}{-\maxdimen} % remove section numbering
\ifLuaTeX
  \usepackage{selnolig}  % disable illegal ligatures
\fi

\begin{document}
\maketitle

\begin{Shaded}
\begin{Highlighting}[]
\FunctionTok{library}\NormalTok{(dplyr)}
\end{Highlighting}
\end{Shaded}

\begin{verbatim}
## 
## Attaching package: 'dplyr'
\end{verbatim}

\begin{verbatim}
## The following objects are masked from 'package:stats':
## 
##     filter, lag
\end{verbatim}

\begin{verbatim}
## The following objects are masked from 'package:base':
## 
##     intersect, setdiff, setequal, union
\end{verbatim}

\begin{Shaded}
\begin{Highlighting}[]
\DocumentationTok{\#\#\#\#\#\#\# Regression lineaire   \#\#\#\#\#\#\#\#\#\#\#\#\#}
\NormalTok{cre}\OtherTok{\textless{}{-}}\FunctionTok{read.csv}\NormalTok{(}\StringTok{"CreditBancaire.csv"}\NormalTok{)}

\DocumentationTok{\#\#\#\#\#\#\#\#\#\#}
\NormalTok{reg\_1}\OtherTok{\textless{}{-}}\FunctionTok{lm}\NormalTok{(}\AttributeTok{data =}\NormalTok{ cre,Jours}\SpecialCharTok{\textasciitilde{}}\NormalTok{Credit}\SpecialCharTok{+}\NormalTok{Type)}

\NormalTok{reg\_1}
\end{Highlighting}
\end{Shaded}

\begin{verbatim}
## 
## Call:
## lm(formula = Jours ~ Credit + Type, data = cre)
## 
## Coefficients:
##      (Intercept)            Credit  TypeConsommation    TypeProduction  
##        8.519e+01        -5.484e-06        -1.486e+01         1.498e+01
\end{verbatim}

\begin{Shaded}
\begin{Highlighting}[]
\NormalTok{reg\_resum}\OtherTok{\textless{}{-}}\FunctionTok{summary}\NormalTok{(reg\_1)}

\NormalTok{reg\_resum  }\CommentTok{\# les resultats detailles}
\end{Highlighting}
\end{Shaded}

\begin{verbatim}
## 
## Call:
## lm(formula = Jours ~ Credit + Type, data = cre)
## 
## Residuals:
##    Min     1Q Median     3Q    Max 
## -96.26 -69.22 -49.06  17.29 270.22 
## 
## Coefficients:
##                    Estimate Std. Error t value Pr(>|t|)    
## (Intercept)       8.519e+01  1.275e+01   6.681 1.63e-10 ***
## Credit           -5.484e-06  1.182e-05  -0.464    0.643    
## TypeConsommation -1.486e+01  1.472e+01  -1.009    0.314    
## TypeProduction    1.498e+01  2.532e+01   0.591    0.555    
## ---
## Signif. codes:  0 '***' 0.001 '**' 0.01 '*' 0.05 '.' 0.1 ' ' 1
## 
## Residual standard error: 104 on 241 degrees of freedom
## Multiple R-squared:  0.007512,   Adjusted R-squared:  -0.004842 
## F-statistic: 0.608 on 3 and 241 DF,  p-value: 0.6104
\end{verbatim}

\begin{Shaded}
\begin{Highlighting}[]
\CommentTok{\# afficher que le tableau des coefficients}
\NormalTok{reg\_resum}\SpecialCharTok{$}\NormalTok{coefficients}
\end{Highlighting}
\end{Shaded}

\begin{verbatim}
##                       Estimate   Std. Error    t value     Pr(>|t|)
## (Intercept)       8.519082e+01 1.275135e+01  6.6809256 1.630050e-10
## Credit           -5.483559e-06 1.182460e-05 -0.4637417 6.432514e-01
## TypeConsommation -1.485726e+01 1.472339e+01 -1.0090929 3.139421e-01
## TypeProduction    1.497503e+01 2.532278e+01  0.5913659 5.548295e-01
\end{verbatim}

\begin{Shaded}
\begin{Highlighting}[]
\CommentTok{\# coefficients et t student}
\NormalTok{reg\_resum}\SpecialCharTok{$}\NormalTok{coefficients[,}\FunctionTok{c}\NormalTok{(}\DecValTok{1}\NormalTok{,}\DecValTok{3}\NormalTok{)]}
\end{Highlighting}
\end{Shaded}

\begin{verbatim}
##                       Estimate    t value
## (Intercept)       8.519082e+01  6.6809256
## Credit           -5.483559e-06 -0.4637417
## TypeConsommation -1.485726e+01 -1.0090929
## TypeProduction    1.497503e+01  0.5913659
\end{verbatim}

\begin{Shaded}
\begin{Highlighting}[]
\DocumentationTok{\#\#\#\#\#\#\# une fonction pour fournir que coeff et t student \#\#\#\#\#\#\#\#\#\#33}
\NormalTok{result}\OtherTok{\textless{}{-}}\ControlFlowTok{function}\NormalTok{(y,x,data)\{}
\NormalTok{  reg}\OtherTok{=}\FunctionTok{lm}\NormalTok{(data[,y]}\SpecialCharTok{\textasciitilde{}}\NormalTok{data[,x])}
\NormalTok{  reg\_details}\OtherTok{\textless{}{-}}\FunctionTok{summary}\NormalTok{(reg)}
\NormalTok{  reg\_details}\SpecialCharTok{$}\NormalTok{coef[,}\FunctionTok{c}\NormalTok{(}\DecValTok{1}\NormalTok{,}\DecValTok{3}\NormalTok{)]}
\NormalTok{\}}


\CommentTok{\# verifier si la fonction marche}
\FunctionTok{result}\NormalTok{(}\StringTok{\textquotesingle{}Jours\textquotesingle{}}\NormalTok{,}\StringTok{\textquotesingle{}Credit\textquotesingle{}}\NormalTok{,}\AttributeTok{data=}\NormalTok{cre)}
\end{Highlighting}
\end{Shaded}

\begin{verbatim}
##                  Estimate    t value
## (Intercept)  8.124861e+01  7.1661463
## data[, x]   -4.770881e-06 -0.4045829
\end{verbatim}

\begin{Shaded}
\begin{Highlighting}[]
\CommentTok{\# moyenne de jours par type de credit}
\NormalTok{cre}\SpecialCharTok{\%\textgreater{}\%}
  \FunctionTok{group\_by}\NormalTok{(Type)}\SpecialCharTok{\%\textgreater{}\%}
  \FunctionTok{summarise}\NormalTok{(}\AttributeTok{Moyenne\_jours=}\FunctionTok{mean}\NormalTok{(Credit))}
\end{Highlighting}
\end{Shaded}

\begin{verbatim}
## # A tibble: 3 x 2
##   Type         Moyenne_jours
##   <chr>                <dbl>
## 1 Commerce           806521.
## 2 Consommation       727061.
## 3 Production         769295.
\end{verbatim}

\begin{Shaded}
\begin{Highlighting}[]
\CommentTok{\# Tests de difference de la moyenne}
\CommentTok{\# verifions si cons=769294 qui est la moyenne du credit en production}
\NormalTok{cons}\OtherTok{\textless{}{-}}\NormalTok{cre}\SpecialCharTok{\%\textgreater{}\%}\FunctionTok{filter}\NormalTok{(Type}\SpecialCharTok{==}\StringTok{\textquotesingle{}Consommation\textquotesingle{}}\NormalTok{)}\SpecialCharTok{\%\textgreater{}\%}\FunctionTok{select}\NormalTok{(Credit)}
\NormalTok{prod}\OtherTok{\textless{}{-}}\NormalTok{cre}\SpecialCharTok{\%\textgreater{}\%}\FunctionTok{filter}\NormalTok{(Type}\SpecialCharTok{==}\StringTok{\textquotesingle{}Production\textquotesingle{}}\NormalTok{)}\SpecialCharTok{\%\textgreater{}\%}\FunctionTok{select}\NormalTok{(Credit)}
\FunctionTok{t.test}\NormalTok{(cons,}\AttributeTok{mu=}\DecValTok{769294}\NormalTok{,}\AttributeTok{alternative =} \StringTok{"two.sided"}\NormalTok{)}
\end{Highlighting}
\end{Shaded}

\begin{verbatim}
## 
##  One Sample t-test
## 
## data:  cons
## t = -0.64345, df = 74, p-value = 0.5219
## alternative hypothesis: true mean is not equal to 769294
## 95 percent confidence interval:
##  596281.5 857841.4
## sample estimates:
## mean of x 
##  727061.4
\end{verbatim}

\begin{Shaded}
\begin{Highlighting}[]
\CommentTok{\#pvalue is = 0.52, than superieur a 0.05 nous acceptons l\textquotesingle{}hypothese nulle}
\CommentTok{\# que cons=769294}


\CommentTok{\# verifions si le credit moyen cons est inferieur a production}
\FunctionTok{t.test}\NormalTok{(cons,prod,}\AttributeTok{alternative =} \StringTok{"less"}\NormalTok{)}
\end{Highlighting}
\end{Shaded}

\begin{verbatim}
## 
##  Welch Two Sample t-test
## 
## data:  cons and prod
## t = -0.26354, df = 25.733, p-value = 0.3971
## alternative hypothesis: true difference in means is less than 0
## 95 percent confidence interval:
##    -Inf 231202
## sample estimates:
## mean of x mean of y 
##  727061.4  769294.8
\end{verbatim}

\begin{Shaded}
\begin{Highlighting}[]
\CommentTok{\# puisque pvalue superieur a 0.05 on accepte l\textquotesingle{}hypothese nuelle}
\CommentTok{\# que cons est sup ou egal a prod}




\CommentTok{\# verifions si cons est superieur a prod}
\FunctionTok{t.test}\NormalTok{(cons,prod,}\AttributeTok{alternative =} \StringTok{\textquotesingle{}greater\textquotesingle{}}\NormalTok{)}
\end{Highlighting}
\end{Shaded}

\begin{verbatim}
## 
##  Welch Two Sample t-test
## 
## data:  cons and prod
## t = -0.26354, df = 25.733, p-value = 0.6029
## alternative hypothesis: true difference in means is greater than 0
## 95 percent confidence interval:
##  -315668.8       Inf
## sample estimates:
## mean of x mean of y 
##  727061.4  769294.8
\end{verbatim}

\begin{Shaded}
\begin{Highlighting}[]
\CommentTok{\#pval=0.6. Donc on accepte H0 que cons est inferieur ou eegal a:}


\CommentTok{\# verifions si cons est egal a prod}
\FunctionTok{t.test}\NormalTok{(cons, prod,}\AttributeTok{alternative =} \StringTok{\textquotesingle{}two.sided\textquotesingle{}}\NormalTok{)}
\end{Highlighting}
\end{Shaded}

\begin{verbatim}
## 
##  Welch Two Sample t-test
## 
## data:  cons and prod
## t = -0.26354, df = 25.733, p-value = 0.7942
## alternative hypothesis: true difference in means is not equal to 0
## 95 percent confidence interval:
##  -371804.9  287338.1
## sample estimates:
## mean of x mean of y 
##  727061.4  769294.8
\end{verbatim}

\begin{Shaded}
\begin{Highlighting}[]
\CommentTok{\#pvalue=0.79 est superieur a 0.05. Donc, on accepte l\textquotesingle{}hypothese nulle}
\CommentTok{\# que cons = prod.}



\DocumentationTok{\#\#\#\#\#\#\#\#\#\#\#\#\#\#\#\#\#\#\#\#\#\#\#\#\#\#\#\#\#\#\#\#\#\#\#}
\DocumentationTok{\#\#\#\# GGPLOT2 \#\#\#\#\#\#\#\#\#}
\DocumentationTok{\#\#\#\#\#\#\#\#\#\#\#\#\#\#\#\#\#\#\#\#\#\#}

\CommentTok{\# nuage de point}
\FunctionTok{library}\NormalTok{(ggplot2)}
\FunctionTok{ggplot}\NormalTok{(}\AttributeTok{data=}\NormalTok{cre,}\FunctionTok{aes}\NormalTok{(}\AttributeTok{x=}\NormalTok{Jours,}\AttributeTok{y=}\NormalTok{Credit))}\SpecialCharTok{+}
  \FunctionTok{geom\_point}\NormalTok{()}
\end{Highlighting}
\end{Shaded}

\includegraphics{regression_files/figure-latex/unnamed-chunk-1-1.pdf}

\begin{Shaded}
\begin{Highlighting}[]
\DocumentationTok{\#\#\#}
\CommentTok{\# size of points}
\FunctionTok{ggplot}\NormalTok{(}\AttributeTok{data=}\NormalTok{cre,}\FunctionTok{aes}\NormalTok{(}\AttributeTok{x=}\NormalTok{Jours,}\AttributeTok{y=}\NormalTok{Credit,}\AttributeTok{color=}\StringTok{\textquotesingle{}red\textquotesingle{}}\NormalTok{))}\SpecialCharTok{+}
  \FunctionTok{geom\_point}\NormalTok{(}\AttributeTok{size=}\FloatTok{2.5}\NormalTok{)}
\end{Highlighting}
\end{Shaded}

\includegraphics{regression_files/figure-latex/unnamed-chunk-1-2.pdf}

\begin{Shaded}
\begin{Highlighting}[]
\DocumentationTok{\#\#}
\CommentTok{\# shape}
\FunctionTok{ggplot}\NormalTok{(}\AttributeTok{data=}\NormalTok{cre,}\FunctionTok{aes}\NormalTok{(}\AttributeTok{x=}\NormalTok{Jours,}\AttributeTok{y=}\NormalTok{Credit,}\AttributeTok{color=}\StringTok{\textquotesingle{}red\textquotesingle{}}\NormalTok{))}\SpecialCharTok{+}
  \FunctionTok{geom\_point}\NormalTok{(}\AttributeTok{size=}\FloatTok{2.5}\NormalTok{,}\AttributeTok{shape=}\DecValTok{6}\NormalTok{)}
\end{Highlighting}
\end{Shaded}

\includegraphics{regression_files/figure-latex/unnamed-chunk-1-3.pdf}

\begin{Shaded}
\begin{Highlighting}[]
\CommentTok{\# find help on shape}
\FunctionTok{help}\NormalTok{(shape,}\StringTok{"ggplot2"}\NormalTok{)}
\end{Highlighting}
\end{Shaded}

\begin{verbatim}
## starting httpd help server ...
\end{verbatim}

\begin{verbatim}
##  done
\end{verbatim}

\begin{Shaded}
\begin{Highlighting}[]
\DocumentationTok{\#\#}
\CommentTok{\#theme}
\FunctionTok{ggplot}\NormalTok{(}\AttributeTok{data=}\NormalTok{cre,}\FunctionTok{aes}\NormalTok{(}\AttributeTok{x=}\NormalTok{Jours,}\AttributeTok{y=}\NormalTok{Credit,}\AttributeTok{color=}\StringTok{\textquotesingle{}red\textquotesingle{}}\NormalTok{))}\SpecialCharTok{+}
  \FunctionTok{geom\_point}\NormalTok{(}\AttributeTok{size=}\FloatTok{2.5}\NormalTok{,}\AttributeTok{shape=}\DecValTok{1}\NormalTok{)}\SpecialCharTok{+}
  \FunctionTok{theme\_bw}\NormalTok{()}
\end{Highlighting}
\end{Shaded}

\includegraphics{regression_files/figure-latex/unnamed-chunk-1-4.pdf}

\begin{Shaded}
\begin{Highlighting}[]
\DocumentationTok{\#\# theme classic}
\FunctionTok{ggplot}\NormalTok{(}\AttributeTok{data=}\NormalTok{cre,}\FunctionTok{aes}\NormalTok{(}\AttributeTok{x=}\NormalTok{Jours,}\AttributeTok{y=}\NormalTok{Credit,}\AttributeTok{color=}\StringTok{\textquotesingle{}red\textquotesingle{}}\NormalTok{))}\SpecialCharTok{+}
  \FunctionTok{geom\_point}\NormalTok{(}\AttributeTok{size=}\FloatTok{2.5}\NormalTok{,}\AttributeTok{shape=}\DecValTok{1}\NormalTok{)}\SpecialCharTok{+}
  \FunctionTok{theme\_classic}\NormalTok{()}
\end{Highlighting}
\end{Shaded}

\includegraphics{regression_files/figure-latex/unnamed-chunk-1-5.pdf}

\begin{Shaded}
\begin{Highlighting}[]
\DocumentationTok{\#\# change axis title}
\FunctionTok{ggplot}\NormalTok{(}\AttributeTok{data=}\NormalTok{cre,}\FunctionTok{aes}\NormalTok{(}\AttributeTok{x=}\NormalTok{Jours,}\AttributeTok{y=}\NormalTok{Credit,}\AttributeTok{color=}\StringTok{\textquotesingle{}red\textquotesingle{}}\NormalTok{))}\SpecialCharTok{+}
  \FunctionTok{geom\_point}\NormalTok{(}\AttributeTok{size=}\FloatTok{2.5}\NormalTok{,}\AttributeTok{shape=}\DecValTok{1}\NormalTok{)}\SpecialCharTok{+}
  \FunctionTok{xlab}\NormalTok{(}\StringTok{"Nombre de jours"}\NormalTok{)}\SpecialCharTok{+}
  \FunctionTok{theme\_classic}\NormalTok{()}
\end{Highlighting}
\end{Shaded}

\includegraphics{regression_files/figure-latex/unnamed-chunk-1-6.pdf}

\begin{Shaded}
\begin{Highlighting}[]
\DocumentationTok{\#\#\# ajouter une tendance}
\FunctionTok{ggplot}\NormalTok{(}\AttributeTok{data=}\NormalTok{cre,}\FunctionTok{aes}\NormalTok{(}\AttributeTok{x=}\NormalTok{Jours,}\AttributeTok{y=}\NormalTok{Credit,}\AttributeTok{color=}\StringTok{\textquotesingle{}red\textquotesingle{}}\NormalTok{))}\SpecialCharTok{+}
  \FunctionTok{geom\_point}\NormalTok{(}\AttributeTok{size=}\FloatTok{2.5}\NormalTok{)}\SpecialCharTok{+}
  \FunctionTok{geom\_smooth}\NormalTok{(}\AttributeTok{color=}\StringTok{\textquotesingle{}black\textquotesingle{}}\NormalTok{)}\SpecialCharTok{+}
  \FunctionTok{xlab}\NormalTok{(}\StringTok{"Nombre de jours"}\NormalTok{)}\SpecialCharTok{+}
  \FunctionTok{theme\_classic}\NormalTok{()}
\end{Highlighting}
\end{Shaded}

\begin{verbatim}
## `geom_smooth()` using method = 'loess' and formula 'y ~ x'
\end{verbatim}

\includegraphics{regression_files/figure-latex/unnamed-chunk-1-7.pdf}

\begin{Shaded}
\begin{Highlighting}[]
\DocumentationTok{\#\#\# enlever la partie achuree et le theme}
\FunctionTok{ggplot}\NormalTok{(}\AttributeTok{data=}\NormalTok{cre,}\FunctionTok{aes}\NormalTok{(}\AttributeTok{x=}\NormalTok{Jours,}\AttributeTok{y=}\NormalTok{Credit,}\AttributeTok{color=}\StringTok{\textquotesingle{}red\textquotesingle{}}\NormalTok{))}\SpecialCharTok{+}
  \FunctionTok{geom\_point}\NormalTok{(}\AttributeTok{size=}\FloatTok{2.5}\NormalTok{)}\SpecialCharTok{+}
  \FunctionTok{geom\_smooth}\NormalTok{(}\AttributeTok{color=}\StringTok{\textquotesingle{}black\textquotesingle{}}\NormalTok{,}\AttributeTok{se=}\ConstantTok{FALSE}\NormalTok{)}\SpecialCharTok{+}
  \FunctionTok{xlab}\NormalTok{(}\StringTok{"Nombre de jours"}\NormalTok{)}\SpecialCharTok{+}
  \FunctionTok{theme\_light}\NormalTok{()}
\end{Highlighting}
\end{Shaded}

\begin{verbatim}
## `geom_smooth()` using method = 'loess' and formula 'y ~ x'
\end{verbatim}

\includegraphics{regression_files/figure-latex/unnamed-chunk-1-8.pdf}

\begin{Shaded}
\begin{Highlighting}[]
\DocumentationTok{\#\#\#\#\#\#\#\#\#\#\#\#\#\#\#\#\#\#\#\#\#\#\#}
\DocumentationTok{\#\#\#\#Geom\_bar  \#\#\#\#\#\#\#\#}
\NormalTok{tsecteur}\OtherTok{\textless{}{-}}\FunctionTok{prop.table}\NormalTok{(}\FunctionTok{table}\NormalTok{(cre}\SpecialCharTok{$}\NormalTok{Type))}
\NormalTok{tsecteur}\OtherTok{=}\FunctionTok{data.frame}\NormalTok{(tsecteur)}
\NormalTok{tsecteur}
\end{Highlighting}
\end{Shaded}

\begin{verbatim}
##           Var1       Freq
## 1     Commerce 0.61632653
## 2 Consommation 0.30612245
## 3   Production 0.07755102
\end{verbatim}

\begin{Shaded}
\begin{Highlighting}[]
\DocumentationTok{\#\#\#}
\FunctionTok{ggplot}\NormalTok{(}\AttributeTok{data=}\NormalTok{tsecteur,}\FunctionTok{aes}\NormalTok{(}\AttributeTok{x=}\NormalTok{Var1,}\AttributeTok{y=}\NormalTok{Freq))}\SpecialCharTok{+}
  \FunctionTok{geom\_bar}\NormalTok{(}\AttributeTok{stat =} \StringTok{\textquotesingle{}identity\textquotesingle{}}\NormalTok{)}
\end{Highlighting}
\end{Shaded}

\includegraphics{regression_files/figure-latex/unnamed-chunk-1-9.pdf}

\begin{Shaded}
\begin{Highlighting}[]
\DocumentationTok{\#\#\#\# remplir les batons de la couleur rouge}
\FunctionTok{ggplot}\NormalTok{(}\AttributeTok{data=}\NormalTok{tsecteur,}\FunctionTok{aes}\NormalTok{(}\AttributeTok{x=}\NormalTok{Var1,}\AttributeTok{y=}\NormalTok{Freq,}\AttributeTok{fill=}\StringTok{\textquotesingle{}red\textquotesingle{}}\NormalTok{))}\SpecialCharTok{+}
  \FunctionTok{geom\_bar}\NormalTok{(}\AttributeTok{stat =} \StringTok{\textquotesingle{}identity\textquotesingle{}}\NormalTok{)}
\end{Highlighting}
\end{Shaded}

\includegraphics{regression_files/figure-latex/unnamed-chunk-1-10.pdf}

\begin{Shaded}
\begin{Highlighting}[]
\DocumentationTok{\#\# changer noms des axes}
\FunctionTok{ggplot}\NormalTok{(}\AttributeTok{data=}\NormalTok{tsecteur,}\FunctionTok{aes}\NormalTok{(}\AttributeTok{x=}\NormalTok{Var1,}\AttributeTok{y=}\NormalTok{Freq,}\AttributeTok{fill=}\StringTok{\textquotesingle{}red\textquotesingle{}}\NormalTok{))}\SpecialCharTok{+}
  \FunctionTok{geom\_bar}\NormalTok{(}\AttributeTok{stat =} \StringTok{\textquotesingle{}identity\textquotesingle{}}\NormalTok{)}\SpecialCharTok{+}\FunctionTok{xlab}\NormalTok{(}\StringTok{"Secteurs"}\NormalTok{)}\SpecialCharTok{+}\FunctionTok{ylab}\NormalTok{(}\StringTok{\textquotesingle{}Proportion\textquotesingle{}}\NormalTok{)}
\end{Highlighting}
\end{Shaded}

\includegraphics{regression_files/figure-latex/unnamed-chunk-1-11.pdf}

\begin{Shaded}
\begin{Highlighting}[]
\DocumentationTok{\#\#\#  A vous de jouer}


\DocumentationTok{\#\#\#\# Graphique en nuage de points avec des groupes}

\FunctionTok{ggplot}\NormalTok{(}\AttributeTok{data=}\NormalTok{cre,}\FunctionTok{aes}\NormalTok{(}\AttributeTok{x=}\NormalTok{Jours,}\AttributeTok{y=}\NormalTok{Credit,}\AttributeTok{color=}\NormalTok{Type))}\SpecialCharTok{+}
  \FunctionTok{geom\_point}\NormalTok{()}\SpecialCharTok{+}
  \FunctionTok{theme\_light}\NormalTok{()}
\end{Highlighting}
\end{Shaded}

\includegraphics{regression_files/figure-latex/unnamed-chunk-1-12.pdf}

\begin{Shaded}
\begin{Highlighting}[]
\DocumentationTok{\#\#\# modifier la taille des points avec size }
\FunctionTok{ggplot}\NormalTok{(}\AttributeTok{data=}\NormalTok{cre,}\FunctionTok{aes}\NormalTok{(}\AttributeTok{x=}\NormalTok{Jours,}\AttributeTok{y=}\NormalTok{Credit,}\AttributeTok{color=}\NormalTok{Type))}\SpecialCharTok{+}
  \FunctionTok{geom\_point}\NormalTok{(}\AttributeTok{size=}\DecValTok{3}\NormalTok{)}\SpecialCharTok{+}
  \FunctionTok{theme\_light}\NormalTok{()}
\end{Highlighting}
\end{Shaded}

\includegraphics{regression_files/figure-latex/unnamed-chunk-1-13.pdf}

\begin{Shaded}
\begin{Highlighting}[]
\DocumentationTok{\#\#\# ajouter une tendance}
\FunctionTok{ggplot}\NormalTok{(}\AttributeTok{data=}\NormalTok{cre,}\FunctionTok{aes}\NormalTok{(}\AttributeTok{x=}\NormalTok{Jours,}\AttributeTok{y=}\NormalTok{Credit,}\AttributeTok{color=}\NormalTok{Type))}\SpecialCharTok{+}
  \FunctionTok{geom\_point}\NormalTok{()}\SpecialCharTok{+}
  \FunctionTok{geom\_smooth}\NormalTok{()}\SpecialCharTok{+}
  \FunctionTok{theme\_light}\NormalTok{()}
\end{Highlighting}
\end{Shaded}

\begin{verbatim}
## `geom_smooth()` using method = 'loess' and formula 'y ~ x'
\end{verbatim}

\includegraphics{regression_files/figure-latex/unnamed-chunk-1-14.pdf}

\begin{Shaded}
\begin{Highlighting}[]
\DocumentationTok{\#\# enlever la partie achuree}
\FunctionTok{ggplot}\NormalTok{(}\AttributeTok{data=}\NormalTok{cre,}\FunctionTok{aes}\NormalTok{(}\AttributeTok{x=}\NormalTok{Jours,}\AttributeTok{y=}\NormalTok{Credit,}\AttributeTok{color=}\NormalTok{Type))}\SpecialCharTok{+}
  \FunctionTok{geom\_point}\NormalTok{()}\SpecialCharTok{+}
  \FunctionTok{geom\_smooth}\NormalTok{(}\AttributeTok{se=}\NormalTok{F)}\SpecialCharTok{+}
  \FunctionTok{theme\_light}\NormalTok{()}
\end{Highlighting}
\end{Shaded}

\begin{verbatim}
## `geom_smooth()` using method = 'loess' and formula 'y ~ x'
\end{verbatim}

\includegraphics{regression_files/figure-latex/unnamed-chunk-1-15.pdf}

\begin{Shaded}
\begin{Highlighting}[]
\DocumentationTok{\#\#\#\#\#\#\#\#\#\#\#\#\#\#\#\#\#\#\#\#\#\#\#}
\DocumentationTok{\#\#\# graphique en ligne}

\FunctionTok{ggplot}\NormalTok{(}\AttributeTok{data=}\NormalTok{cre,}\FunctionTok{aes}\NormalTok{(}\AttributeTok{x=}\NormalTok{Jours,}\AttributeTok{y=}\NormalTok{Credit))}\SpecialCharTok{+}
  \FunctionTok{geom\_line}\NormalTok{()}\SpecialCharTok{+}
  \FunctionTok{theme\_classic}\NormalTok{()}
\end{Highlighting}
\end{Shaded}

\includegraphics{regression_files/figure-latex/unnamed-chunk-1-16.pdf}

\begin{Shaded}
\begin{Highlighting}[]
\DocumentationTok{\#\# change line color}
\FunctionTok{ggplot}\NormalTok{(}\AttributeTok{data=}\NormalTok{cre,}\FunctionTok{aes}\NormalTok{(}\AttributeTok{x=}\NormalTok{Jours,}\AttributeTok{y=}\NormalTok{Credit))}\SpecialCharTok{+}
  \FunctionTok{geom\_line}\NormalTok{(}\AttributeTok{col=}\StringTok{\textquotesingle{}navy\textquotesingle{}}\NormalTok{)}\SpecialCharTok{+}
  \FunctionTok{theme\_classic}\NormalTok{()}
\end{Highlighting}
\end{Shaded}

\includegraphics{regression_files/figure-latex/unnamed-chunk-1-17.pdf}

\begin{Shaded}
\begin{Highlighting}[]
\DocumentationTok{\#\#\#\#\#}
\FunctionTok{ggplot}\NormalTok{(}\AttributeTok{data =}\NormalTok{ cre,}\FunctionTok{aes}\NormalTok{(Jours))}\SpecialCharTok{+}
  \FunctionTok{geom\_histogram}\NormalTok{(}\AttributeTok{bins=}\DecValTok{10}\NormalTok{)}\SpecialCharTok{+}
  \FunctionTok{theme\_light}\NormalTok{()}
\end{Highlighting}
\end{Shaded}

\includegraphics{regression_files/figure-latex/unnamed-chunk-1-18.pdf}

\end{document}
